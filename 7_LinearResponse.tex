\documentclass{article}

\usepackage{lipsum}
\usepackage[margin=1.6in,includefoot]{geometry}

\usepackage{amsmath}
\usepackage{bbm}

% header and footer stuff
\usepackage{fancyhdr}
\pagestyle{fancy}
%\fancyhead{}
\fancyfoot{}
\fancyfoot[R]{\thepage\ }
\renewcommand{\headrulewidth}{0pt}
%%%%%%%EMANUELE%%%%%%%%
\newcommand*\diff{\mathop{}\!\mathrm{d}}
\newcommand*\Diff[1]{\mathop{}\!\mathrm{d^#1}}
\newcommand*\Tder[1]{\mathop{}\!\frac{\diff #1}{\diff \mathrm{t}}}
\newcommand*\tder[1]{\mathop{}\!\frac{\partial #1}{\partial \mathrm{t}} }
\newcommand*\mean[1]{\mathop{}\!\langle #1 \rangle}
 %%%%%%%%%%%%%%%%%%%%%%%
 
\begin{document}

\begin{titlepage}
	\begin{center}
	
	\line(1,0){300}\\
	[5mm]
	\huge{\bfseries Linear Response Theory}\\
	[2mm]
	\line(1,0){200}\\
	[2cm]
	\textsc{\Large Meccanica statistica del disequilibrio: fondamenti e applicazioni} \\
	[8cm]
	
	\end{center}
	
	\begin{flushright}
	\textsc{\LARGE Jacopo Pasqualini}\\
	[0.5cm]
	\textsc{\large Università degli studi di Torino\\
	[0.5cm]
	A.A. 2019/2020 }
	\end{flushright}
	
\end{titlepage}

\section{Linear Response Theory}\label{sec:langapp}

Consider a perturbed hamiltonian system with:

$$H(P,Q) = H_0(P,Q) + \lambda A(P,Q) \quad \textit{, with: } (P,Q) = (p_1,...,p_N,q_1,...q_N) = \Gamma$$

Where the perturbation $\lambda A$ is and remains small. We assume that the equilibrium with $H_0$ turn into the equilibrium with $H$. If the equilibrium probability distribution is canonical, to first order we write:
\begin{align*}
 f(\Gamma) &= \frac{ e^{-\beta H(\Gamma)} }{ \int d \Gamma e^{-\beta H(\Gamma)}} = \frac{ e^{-\beta H_0(\Gamma)} e^{-\beta \lambda A(\Gamma)} }{  \int d \Gamma e^{-\beta H_0(\Gamma)} e^{-\beta \lambda A(\Gamma)}} = \\
  &= \frac{ e^{-\beta H_0(\Gamma)}[ 1 - \beta \lambda A(\Gamma) + \mathcal{O}( (\beta \lambda A)^2) ] }{ \int d \Gamma e^{-\beta H_0(\Gamma)}[ 1 - \beta \lambda A(\Gamma) + \mathcal{O}( (\beta \lambda A)^2) ]} =  \\
   &\approx \frac{ e^{-\beta H_0(\Gamma)} }{ \int d \Gamma e^{-\beta H_0(\Gamma)} } \frac{1-\lambda \beta A(\Gamma) }{ 1-\lambda \beta  \langle A(\Gamma) \rangle_0} = f_0(\Gamma) [ 1 - \lambda \beta A(\Gamma) ][ 1 + \lambda \beta \langle A \rangle_0 + \mathcal{O}( (\beta \lambda A)^2) ]    
\end{align*}

i.e.

$$ f(\Gamma) \approx f_0(\Gamma)[1 - \lambda \beta \{ A(\Gamma) - \langle A \rangle_0  \}]$$
In the third line we approximated the expression for $f(\Gamma)$ to first order in $\lambda$, then we evaluated the term $\mean{A(\Gamma)}_0$. 
Here $f_0$, $\mean{}_0$ denote unperturbed canonical quantities. We are assuming that time averages equal ensemble averages in both equilibrium states: who says that the perturbed system will relax to equilibrium? If the perturbed system reaches the equilibrium this is canonical. 

What is the response to the perturbation? 

If $B : \mathcal{M} \mapsto \mathbbm{R} $ is a generic phase space function, one can write:
\begin{align*}
\langle \Delta B \rangle &= \int d \Gamma B(\Gamma) [f(\Gamma) - f_0(\Gamma) ] \approx - \lambda \beta \int d \Gamma B(\Gamma) [ A(\Gamma) - \langle A \rangle_0 ] f_0(\Gamma) =\\ &= - \lambda \beta [\langle A B \rangle_0 - \langle A \rangle_0 \langle B \rangle_0]    
\end{align*}

That is the unperturbed correlation function of $A$ and $B$. \newline

\textbf{NOTE:} There is no dynamics at all here: $A$, $B$, $f_0$, $f$ are phase function. Which shows that the response of $B$ is related to its correlation with the perturbation $A$. In particular, taking $B=A=H_0$, one get:

\begin{equation}
\frac{ \langle \Delta H_0 \rangle }{ - \lambda \beta } \approx \langle {H_0}^2 \rangle_0 - \langle H_0 {\rangle_0}^2 = k_B T^2 c_V
\end{equation}

This equation is exact in the $\lambda \to 0$ limit and this links the response to an energy perturbation, or the heat capacity, to the energy fluctuations, i.e. the variance of the hamiltonian $\sigma^2(H_0)$. It's very important to note that macroscopic measurement reveals microscopic fluctuations.\newline



More in general, let $H(\Gamma,t)=H_0(\Gamma) - \mathcal{I}(t) A(\Gamma)$ and let us study the evolution in time (whether $\mathcal{I}$ is stochastic or not): first of all we have:

$$\frac{d q_j}{ dt } = \frac{\partial H_0}{ \partial p_j} - \mathcal{I}(t) \frac{\partial A}{\partial p_j} = \frac{\partial H_0}{ \partial p_j} - \mathcal{I}(t) K^q_j \quad (K^q_j = \frac{\partial A}{\partial p_j})$$

$$\frac{d p_j}{ dt } = - \frac{\partial H_0}{ \partial q_j} + \mathcal{I}(t) \frac{\partial A}{\partial q_j} = - \frac{\partial H_0}{ \partial q_j} - \mathcal{I}(t) K^q_j \quad (K_j^p = - \frac{\partial A}{\partial q_j})$$

Then, introduce the operators $\mathcal{L}_0$, $\mathcal{L}_{ext}$ as:

$$ i \mathcal{L}_0 f = \{ f,H_0 \} = \sum_{j=1}^{N} (\frac{\partial H_0}{\partial p_j} \frac{\partial}{\partial q_j} - \frac{\partial H_0}{\partial p_j}  \frac{\partial}{\partial p_j} ) f$$

where the coordinates of this operator, $\Gamma_0$, are the ones of the unperturbed system, and

$$i \mathcal{L}_{ext} f = -\mathcal{I}(t) \{ f , A \} = -\mathcal{I}(t)  \sum_{j=1}^{N} (K_j^q \frac{\partial}{\partial q_j} + K_j^p \frac{\partial}{\partial p_j}) f$$

In order to study the evolution of $f$, starting from a canonical $f_0$ a temperature $T$,

$$ f_0(\Gamma)= \frac{1}{\mathcal{Z}(\beta)} e^{-\beta H_0(\Gamma)} = f_{eq}(\Gamma)$$

which is invariant for the equilibrium dynamics determined by $H_0$: $i \mathcal{L}_0 f_{eq} = 0 $

The solution of the entire Liouville equation with $\mathcal{L} = \mathcal{L}_0 + \mathcal{L}_{ext}(t)$ can be obtained as follows, considering that $ \mathcal{L}_0$ is constant in time:

$$ \frac{ f_t(\Gamma) - f_{t-\Delta t}(\Gamma)}{\Delta t} = -i(\mathcal{L}_0 f_{t-\Delta t} (\Gamma) +  \mathcal{L}_{ext}(t-\Delta t) f_{t-\Delta t} (\Gamma) ) + \mathcal{O}(\Delta t) $$

\begin{align*}
f_t&= f_{t-\Delta t} - \mathcal{L}_0 f_{t-\Delta t} \Delta t -  i \mathcal{L}_{ext}(t-\Delta t) f_{t-\Delta t} \Delta t + \mathcal{O}(\Delta t^2)\\
&=e^{-i \Delta t \mathcal{L}_0}  f_{t-\Delta t}  - i \mathcal{L}_{ext}(t-\Delta t) f_{t-\Delta t} \Delta t + \mathcal{O}(\Delta t^2) \\
&= e^{-i \Delta t \mathcal{L}_0}  f_{t-\Delta t}  -i e^{-i(t-t+0) \mathcal{L}_0} \mathcal{L}_{ext}(t-\Delta t) f_{t-\Delta t} \Delta t + \mathcal{O}(\Delta t^2) \\
&= e^{-i \Delta t \mathcal{L}_0}  [f_{t- 2\Delta t} - i \mathcal{L}_0 f_{t- 2\Delta t} \Delta t - i \mathcal{L}_{ext}(t- 2\Delta t) f_{t- 2\Delta t} \Delta t ] - i \Delta t e^{i 0 \mathcal{L}_0} \mathcal{L}_{ext}(t-\Delta t) f_{t-\Delta t} + \mathcal{O}(\Delta t^2)\\
&= e^{- 2i \Delta t \mathcal{L}_0} f_{t- 2 \Delta t} - i  \Delta t e^{-i \Delta t \mathcal{L}_0} \mathcal{L}_{ext}(t-2\Delta t) f_{t-2 \Delta t} - i \Delta t e^{- i 0 \mathcal{L}_0} \mathcal{L}_{ext}(t-\Delta t) f_{t - \Delta t} + \mathcal{O}(\Delta t^2) \\
&= e^{- 2i \Delta t \mathcal{L}_0} f_{t- 2 \Delta t} - i  \Delta t e^{-i(t-t + \Delta t )\mathcal{L}_0} \mathcal{L}_{ext}(t-2\Delta t) f_{t-2 \Delta t} - i \Delta t e^{- i (t-t+0) \mathcal{L}_0} \mathcal{L}_{ext}(t-\Delta t) f_{t - \Delta t} + \mathcal{O}(\Delta t^2) \\
& = ...
\end{align*}

continuing we get:

$$ ... = e^{-i n \Delta t \mathcal{L}_0} f_{t - n\Delta t} - i \Delta t \sum_{k=1}^{n} e^{-i(k-1) \Delta t \mathcal{L}_0 } \mathcal{L}_{ext}(t-k \Delta t) f_{t-k \Delta t} + \mathcal{0}(\Delta t^2) $$

taking $t = n \Delta t$, $t_k = k \Delta t, t-t_k = {t'}_k$, we get $(k-1)\Delta t= t - {t'}_{k-1}$ and

$$f_t(\Gamma) = e^{-i t \mathcal{L}_0} f_0(\Gamma) - i \Delta t \sum_{k=1}^{n} e^{-i (t-{t'}_{k-1}) \mathcal{L}_0 } \mathcal{L}_{ext}({t'}_k) f_{{t'}_k}(\Gamma) + \mathcal{O}(\Delta t^2) $$


therefore the solution of the Liouville equation $\partial_t f = -i(\mathcal{L}_0 + \mathcal{L}_ {ext}(t)) f$ can be expressed as:

$$ f_t(\Gamma) = e^{-i t \mathcal{L}_0} f_{eq}(\Gamma) - i \int_{0}^{t} dt^{'} e^{-i(t-t') \mathcal{L}_0} \mathcal{L}_{ext}(t') f_{t'}(\Gamma)$$


which, neglecting products of $\mathcal{L}_{ext}$ leads to:

$$ f_t(\Gamma) = f_{eq}(\Gamma) - i \int_{0}^{t} dt^{'} e^{-i(t-t') \mathcal{L}_0} \mathcal{L}_{ext}(t') f_{t'}(\Gamma) + \textit{ higher order in } \mathcal{L}_{ext}$$


Consequently, letting $\langle B \rangle_t = \int d \Gamma B(\Gamma) f_t(\Gamma)$, $\langle B \rangle_{eq} =  \int d \Gamma B(\Gamma) f_{eq}(\Gamma) $, we have:

\begin{align*}
\langle \Delta B \rangle_t  &= \langle B \rangle_t - \langle B \rangle_{eq} =\\
& = -i \int d \Gamma B(\Gamma) \int_{0}^{t} dt' e^{-i(t-t') \mathcal{L}_0} \mathcal{L}_{ext}(t') f_{eq}(\Gamma) \\
& =  -i \int d \Gamma B(\Gamma) \int_{0}^{t} dt' e^{-i(t-t') \mathcal{L}_0} i \mathcal{I}(t') \{ f_{eq}, A \}(\Gamma) \\
& = \int d \Gamma B(\Gamma)  \int_{0}^{t} dt' e^{-i(t-t') \mathcal{L}_0} \mathcal{I}(t') \{ f_{eq}, A \}(\Gamma)
\end{align*}

which is first order in the field. Consider now that $f_{eq}$ depends implicitly on $\Gamma$, since it dependes explicitly only on $H_0$:

$$\{ f_{eq}, A \} = \frac{\partial f_{eq}}{\partial Q}  \frac{\partial A}{\partial P} -  \frac{\partial f_{eq}}{\partial P} \frac{\partial A}{\partial Q} =  \frac{\partial f_{eq} }{\partial H_0} \frac{\partial H_0 }{\partial Q} \frac{\partial A}{\partial P} -  \frac{\partial f_{eq}}{\partial H_0}  \frac{\partial H_0}{\partial P}  \frac{\partial A}{\partial Q} =  \frac{\partial f_{eq} }{\partial H_0} \{ H_0, A \}$$

where 
$$\frac{\partial f_{eq} }{\partial H_0} = - \beta f_{eq}$$
$$ \{ H_0, A \} = -  \{ A, H_0 \} =  -  \{ A, H_0 + \mathcal{I}(t) A \} = - \{ A, H \} = -\frac{dA}{dt} \rightarrow \{ f_{eq}, A \} = \beta f_{eq} \dot{A}$$

since $\dot{A}_{unperturbed} = \dot{A}_{perturbed}$ we obtain:

\begin{equation}
	\langle \Delta B \rangle_t = \beta \int_{0}^{t} dt' \mathcal{I}(t') \int d \Gamma B(\Gamma) e^{-i(t-t')\mathcal{L}_0 } \dot{A}(\Gamma) f_{eq}(\Gamma)
\end{equation}

because of the unitarity of the  Liouville operator, the time evolution can be moved from density to observable as follows:

\begin{equation}
	\langle \Delta B \rangle_t = \beta \int_{0}^{t} dt' \mathcal{I}(t') \int d \Gamma  \dot{A}(\Gamma) f_{eq}(\Gamma) e^{i(t-t')\mathcal{L}_0 }B(\Gamma)
\end{equation}

then we can write:

$$ \langle \Delta B \rangle_t  = \int_{0}^{t} dt' R(t-t') \mathcal{I}(t') $$

with the response function defined by:

\begin{equation}
R(t-t') = \beta \int d \Gamma  \dot{A}(\Gamma) f_{eq}(\Gamma) e^{i(t-t')\mathcal{L}_0 }B(\Gamma) = \beta \langle \dot{A} (B \cdot \Phi^t) \rangle_{eq}
\end{equation}


The former equations show that even linear response is affected by memory effects, hence the markovian behaviour appear as an idealization, possibily justified by lack of sensitivity of our in measurements.


\newpage

This non-markovian behaviour switches on immediately when $\mathcal{I} \neq 0$ hence conceptually distinguishes all non-equilibrium phenomena, no matter how close to equilibrium from equilibrium phenomena (some observables are very sensitive to that as less).

Memory effects are quite hard to tame because they imply that all history hears on the present state, and this includes the initial state, which may be arbitrarily prepared.

This is one point which evidences why progresses in non-equilibrium statistical mechanics has proved so much harder than in equilibrium.

The difficulty emerges because statistical mechanics adopts a microscopic perspective, where many other issues may look exceedingly complicated.

From a macroscopic point of view, things look easier. For instance, the memory effects imply that all non-equilibrium fluids are viscoelastic. In practice, however, these effects become relevant only when the system is exceedingly far from equilibrium of (
\emph{almost equivalently}) when the system is not macroscopic.

The microscopic approach is then necessary and one must reconcile it with the macroscopic description. Most often, this amounts to make assumptions about the microscopic dynamic which look obvious and natural but, in fact, are seldom  in small systems or very far from equilibrium. This makes them very hard, if not impossible, to prove.

Consider now the quantity

$$\{ A, H_0 \} = \sum_{j=1}^{N}  \frac{\partial A}{\partial q_j}  \frac{\partial H_0}{\partial p_j} -  \frac{\partial A}{\partial p_j}  \frac{\partial H_0}{\partial q_j} = -  \sum_{j=1}^{N}\frac{\partial H_0}{\partial p_j} K^p_j - F^{(0)}_j K^q_j  $$

where $ F^{(0)}_j =  \frac{\partial H_0}{\partial q_j}$ is the internal force. If $\mathcal{I}(t) A(\Gamma)$ has units of energy and $\mathcal{I}$ of force, $A$ has units of lenght.Then $K^p_j$ are pure numbers and $ \frac{\partial H_0}{\partial p_j} K^p_j$ is a velocity flow which has $m/sec$ units. The same holds for $ F^{(0)}_j K^q_j$. Therefore $\{ A, H_0\}$ is the flow due to external driving of the

$$ \textit{dissipative flux} \quad J(\Gamma) = -\{ A, H_0\}(\Gamma)= -\frac{dA}{dt}(\Gamma)$$

where the "-" sign is clear if one considers the case with $\mathcal{I}(t) A(\Gamma) = -F_j q_j$, so that particles experience an external force F (in positive direction), and $K^q_j=0$, so $\{ A, H_0 \} = - \sum_{j=1}^{N} \frac{\partial H_0}{\partial p_j}$, which is opposite to the current. This leads to

$$ R(t) = -\beta \langle J (B \circ \Phi^t ) \rangle_{eq} $$

This theory is practically complete, but meets some difficulty. If the external force where constant, F say, the purely hamiltonian evolution would have no steady state, hence the linear susceptibilities R would not be enstablished, R would have no limit.

The problem is solved in non equilibrium molecular dynamics by introduction of thermostatting therms in the equations of motion.

On the other hand, although there are practical difficulties, the theory is correct in the $F \to 0$ limit and in that limit holds for all times, and states that the response con be computed from ensembles of trajectories, weighted with the known equilibrium probability.

Then one expects thet what holds for ensembles holds for single system as well. Tipically, this is the case, this is the case for systems of many particles, but which microscopic conditions lead to this result is not clear. For sure, particle must interact, but how?

And it is relevant for non macroscopic systems. It has been neglected in the past for obvious reasons. We are now in conditions to relate equilibrium to non-equilibrium properties, this is made through  the \emph{ Green-Kubo relations} for transport coefficients.

\subsection{Diffusion with constant Force}

In the case of diffusion one applies a spatially uniform force ($\mathcal{I}(t)=1$) along x, to all particles. According to the theory, this leads to

$$ \mathcal{I}(t) A(\Gamma) = -F \sum_{j=1}^{N} q^x_j$$

and

$$ J(\Gamma) = - \{ -F  \sum_{j=1}^{N} q^x_j , H_0 \} = \sum_{j=1}^{N} F \frac{\partial H_0}{\partial p_j} = \frac{F}{m} \sum_{j=1}^{N} p^x_j $$


take $B(\Gamma) = U_x(F) = \frac{1}{m}  \sum_{j=1}^{N} p^x_j \rightarrow \langle U_x \rangle_{eq} = 0 $

\begin{align*}
 \langle \Delta U_x \rangle_t & = \beta \int_{0}^{t} dt' F \int d \Gamma f_{eq}(\Gamma) \{ A, H_0 \}(\Gamma ) e^{i(t-t')\mathcal{L}_0} B(\Gamma) \\
 & = \beta \int_{0}^{t} dt' \langle \dot{A}(0) B(t-t') \rangle_{eq} \\
 & = \beta \int_{0}^{t} dt'F \langle  \sum_{j=1}^{N} \dot{q}^x_j \sum_{k=1}^{N} \frac{1}{m} p^x_k(t-t')\rangle_{eq} \\
 & = \frac{\beta F}{ m^2} \int_{0}^{t} dt' \sum_{j,k=1}^{N} \langle p^x_j(0) p^x_k(t-t')\rangle_{eq}
\end{align*}

Now, in the equilibrium state, particles velocities are not correlated, hence the sum of the mean of the cross products vanishes, and one obtains:

\begin{equation}
 \langle \Delta U_x \rangle_t =  \frac{\beta F}{ m^2} \int_{0}^{t'} dt' \sum_{j=1}^{N} \langle p^x_j(0) p^x_j(t-t')  \rangle_{eq}
\end{equation}

setting $s = t -t'$, $ds = dt'$

$$ = \frac{\beta F}{ m^2} \int_{t}^{0} (- ds)  \sum_{j=1}^{N} \langle p^x_j(0) p^x_j(s)  \rangle_{eq} = $$

$$  =  \frac{\beta F}{ m^2} \int_{0}^{t} ds  \sum_{j=1}^{N} \langle p^x_j(0) p^x_j(s)  \rangle_{eq} = $$

As mobility is defined by $  \langle \Delta U_x \rangle_{\infty} = \mu F $ we obtain

(Remark: $p^x_j(0) p^x_j(t) =  p^x_j (p^x_j \circ \Phi^t )$ is the dissipative flux times evolved observable )

\begin{equation}
\mu =  \frac{\beta}{ m^2} \int_{0}^{\infty} dt  \sum_{j=1}^{N} \langle p^x_j(0) p^x_j(t)  \rangle_{eq} 
\end{equation}

Which is called \emph{Green-Kubo relation} for $\mu$.

Is shows a striking correlation between equilibrium properties (the 2-time correlation function) and non-equilibrium ones ($\mu$). The diffusion coefficient is obtained via Einstein's relation $D = \mu / \beta = k_B T \mu$

\subsection{Shear Flow in 2 dimensions}

Here we take $\mathcal{I} = 1$ and $A(\Gamma)= \frac{\gamma}{2} \sum_{j=1}^{N} q^y_j p^x_j + q^x_j p^y_j$, where $\gamma = \frac{\partial v_x}{\partial y}$ is the shear rate.

Then one obtains the so-called \emph{Slodd equations}

\begin{eqnarray*}
 \frac{q^x_j}{\partial t} = \frac{p^x_j}{m} + \gamma q^y_j &\quad&  \frac{q^y_j}{\partial t} = \frac{p^y_j}{m} + \gamma q^x_j \\
 \frac{p^x_j}{\partial t} =  F^x_j - \gamma p^y_j &\quad&  \frac{p^y_j}{\partial t} = F^y_j - \gamma p^x_j
\end{eqnarray*}

\begin{equation}
J(\Gamma) = - \{ A, H_0 \} = -\frac{\gamma}{2} \sum_{j=1}^{N} \big [ p^x_j \frac{\partial H_0}{\partial p^y_j} + p^y_j \frac{\partial H_0}{\partial p^x_j} - q^y_j \frac{\partial H_0}{\partial q^x_j} - q^x_j \frac{\partial H_0}{\partial q^y_j}  \big]
\end{equation}

$$ = -\frac{\gamma}{2} \sum_{j=1}^{N} \frac{2}{m} p^x_j p^y_j + q^y_j F^x_j + q^x_j F^y_j $$

i.e. the dissipative flux is proportional to the off diagonal element of the pressure tensor:

$$ P^{x,y} = \frac{1}{2V}  \sum_{j=1}^{N} \frac{2}{m} p^x_j p^y_j + q^y_j F^x_j + q^x_j F^y_j= - \frac{1}{\gamma V} J $$

then

\begin{equation}
\langle \Delta P^{x,y} \rangle_t = - \int_{0}^{t} \beta \gamma V \langle P^{x,y}(0) P^{x,y}(t) \rangle_{eq} dt'
\end{equation}

and the shear viscosity takes the form of another Green-Kubo relation (with $\langle P^{x,y} \rangle_0 $):

\begin{equation}
\eta = - lim_{t \to \infty} \frac{\langle \Delta P^{x,y} \rangle_t}{\gamma} = \beta V \int_{0}^{t} \langle P^{x,y}(0) P^{x,y}(t) \rangle_{eq} dt'
\end{equation}

\subsection{Thermal Conductivity}:

\begin{equation}
S =  \sum_{j=1}^{N} ( \frac{p^2_j}{2m} - \langle \epsilon_j \rangle ) \frac{p_j}{m} + \frac{1}{2}  \sum_{i=1}^{N}  \sum_{j \neq i = 1}^{N} (q_{i,j}F_{i,j} + U(q_{i,j}) ) \frac{1}{m}
\end{equation}

where $\langle \epsilon_j \rangle$ is the enthalpy per particle, U is the prior particle potential and $q_{i,j} = q_i - q_j$

so one obtains:

\begin{equation}
\lambda = \frac{1}{3 V k_B T^2} \int_{0}{\infty} \langle S(0) S(t) \rangle_{eq}
\end{equation} \newline

\subsection{Onsager Regression Hypothesis}

The linear theory may be considered complete with the addition of this hypothesis, which states:

$$ \emph{The average regression of fluctuations will obey the same laws as the corresponding macroscopic irreversible process} $$

in other words, if a system finds itself in a stat other than its (attracting) steady state (he meant equilibrium state), it will return to it in a way which does not depend on how it got away. So relation after a spontaneous fluctuation or relaxation from a given initial condition are the same. The FDR justifies this statement.
Consider the observable $\alpha(\Gamma)$, whose average equilibrium value is $\alpha_0$

$$ \alpha_0 = \langle \alpha \rangle = \frac{\int d \Gamma e^{- \beta H(\Gamma) } \alpha(\Gamma) }{\int d \Gamma e^{- \beta H(\Gamma) } } $$

Let $\delta \alpha(t) = \alpha(t) - \alpha_0 $

The regression hypothesis takes then the form:

\begin{equation}
\frac{ \langle \alpha \rangle_t - \alpha_0 }{ \langle \alpha \rangle_0 - \alpha_0 } = \frac{ \langle \delta \alpha (t) \langle \delta \alpha (0) \rangle}{ \langle \delta \alpha(0)^2 \rangle } = \frac{ \langle \alpha(t) \alpha(0) \rangle_t - \alpha_0^2 }{ \langle \alpha^2 \rangle_0 - \alpha_0^2 }
\end{equation}

In the linear regime, consider $H + Delta H $ as the unperturbed hamiltonian and $-\Delta H$ as the perturbation:

$$ \langle \alpha \rangle_t = \langle \alpha \rangle_{\Delta H} + \beta F \int_{0}^{t} dt' \langle \alpha(0) \dot{\alpha}(t-t') \rangle_{\Delta H} =  \langle \alpha \rangle_{\Delta H} + \beta F [ \langle \alpha(t) \alpha(0) \rangle_{\Delta H} -  \langle \alpha^2 \rangle_{\Delta H} ] $$

At $t=0$, one has $\langle \alpha \rangle_{\Delta H} = \alpha_0 + \delta \alpha(0)$ and for $t \to \infty $, $\langle \alpha \rangle \to \alpha_0$, $\langle \alpha(t) \alpha(0) \rangle_{\Delta H} \to \langle \alpha \rangle^2_{\Delta H}$ so, if $\Delta H$ can be neglected in $\langle \alpha \rangle_{\Delta H} + \beta F [ \langle \alpha(t) \alpha(0) \rangle_{\Delta H} -  \langle \alpha^2 \rangle_{\Delta H} ]$ in the linear regime:

$$ \langle \alpha \rangle_t = \alpha_0 + \beta F [ \langle \alpha(t) \alpha(0) \langle ] - \langle \alpha \rangle^2 $$

hence the regression hypothesis is a particular fluctuation dissipation direction.

From a microscopic point of view, indeed, a perturbation amounts to a new evolution, whose trajectory in phase space differs by an amount, which linear response assumes to be

$$ \delta x_i(t) = \sum_{j} \frac{ \partial x_i(t)}{ \partial x_j(0)} \delta x_j(0) + o(|\delta x(0)|^2) $$

where $\delta x(0)$ is the initial perturbation. But trajectories separate exponentially fast, $ \frac{ \partial x_i(t)}{ \partial x_j(0)} \approx e^{\lambda t}$ where $\lambda geq 0$ is the Lyapunov exponent, therefore the first order expansion loses its meaning in the time order: $\tau = \frac{1}{\lambda} \ln (\frac{L}{\delta x(0)})$ where $L$ is the size of a typical spontaneous fluctuation of x. One obtains that linear response holds one second in an electric conductor if its electrons are subjected to a field smaller than $10^{-20} V/m$, in disagreement with the much larger range of velocity of Ohm's las.
The answer to this objection is that:

1) linear response concerns averages and not single trajectories and, typically, averages are more stable in chaotic than non-chaotic systems.

2) the time $t$ needed for the GK formula e.g.

$$ \sigma \approx \beta V \int_{0}{t} \langle J(t) J(0) \rangle_0 dt $$

to approximate $\sigma$ can be estimated to be very short, i.e. to be mesoscopic. Once it has converged, the remaining time ".." keys looking at fluctuations which does not change the result. So, if one requires that things go well for $10^{-10}$ seconds sinche what follows does not matcer??, one can use fields up to $10^4 V/m$




\end{document}